\documentclass{article}

% The automated optical recognition software used to digitize resume
% information works best with fonts that do not have serifs. This
% command uses a sans serif font throughout. Uncomment both lines (or at
% least the second) to restore a Roman font (i.e., a font with serifs).
%\usepackage{times}
%\renewcommand{\familydefault}{\sfdefault}

% This is a helpful package that puts math inside length specifications
\usepackage{calc}
\usepackage{comment}

% Simpler bibsection for CV sections
% (thanks to natbib for inspiration)
\makeatletter
\newlength{\bibhang}
\setlength{\bibhang}{1em} %1em}
\newlength{\bibsep}
 {\@listi \global\bibsep\itemsep \global\advance\bibsep by\parsep}
\newenvironment{bibsection}%
        {\begin{enumerate}{}{%
%        {\begin{list}{}{%
       \setlength{\leftmargin}{\bibhang}%
       \setlength{\itemindent}{-\leftmargin}%
       \setlength{\itemsep}{\bibsep}%
       \setlength{\parsep}{\z@}%
        \setlength{\partopsep}{0pt}%
        \setlength{\topsep}{0pt}}}
        {\end{enumerate}\vspace{-.6\baselineskip}}
%        {\end{list}\vspace{-.6\baselineskip}}
\makeatother

% Layout: Puts the section titles on left side of page
\reversemarginpar

%
%         PAPER SIZE, PAGE NUMBER, AND DOCUMENT LAYOUT NOTES:
%
% The next \usepackage line changes the layout for CV style section
% headings as marginal notes. It also sets up the paper size as either
% letter or A4. By default, letter was used. If A4 paper is desired,
% comment out the letterpaper lines and uncomment the a4paper lines.
%
% As you can see, the margin widths and section title widths can be
% easily adjusted.
%
% ALSO: Notice that the includefoot option can be commented OUT in order
% to put the PAGE NUMBER *IN* the bottom margin. This will make the
% effective text area larger.
%
% IF YOU WISH TO REMOVE THE ``of LASTPAGE'' next to each page number,
% see the note about the +LP and -LP lines below. Comment out the +LP
% and uncomment the -LP.
%
% IF YOU WISH TO REMOVE PAGE NUMBERS, be sure that the includefoot line
% is uncommented and ALSO uncomment the \pagestyle{empty} a few lines
% below.
%

%% Use these lines for letter-sized paper
\usepackage[paper=letterpaper,
            %includefoot, % Uncomment to put page number above margin
            marginparwidth=1.2in,     % Length of section titles
            marginparsep=.05in,       % Space between titles and text
            margin=.5in,               % 1 inch margins
            includemp]{geometry}

%% Use these lines for A4-sized paper
%\usepackage[paper=a4paper,
%            %includefoot, % Uncomment to put page number above margin
%            marginparwidth=30.5mm,    % Length of section titles
%            marginparsep=1.5mm,       % Space between titles and text
%            margin=25mm,              % 25mm margins
%            includemp]{geometry}

%% More layout: Get rid of indenting throughout entire document
\setlength{\parindent}{0in}

\usepackage[shortlabels]{enumitem}

%% Reference the last page in the page number
%
% NOTE: comment the +LP line and uncomment the -LP line to have page
%       numbers without the ``of ##'' last page reference)
%
% NOTE: uncomment the \pagestyle{empty} line to get rid of all page
%       numbers (make sure includefoot is commented out above)
%
\usepackage{fancyhdr,lastpage}
\pagestyle{fancy}
\pagestyle{empty}      % Uncomment this to get rid of page numbers
\fancyhf{}\renewcommand{\headrulewidth}{0pt}
\fancyfootoffset{\marginparsep+\marginparwidth}
\newlength{\footpageshift}
\setlength{\footpageshift}
          {0.5\textwidth+0.5\marginparsep+0.5\marginparwidth-2in}
\lfoot{\hspace{\footpageshift}%
       \parbox{4in}{\, \hfill %
                    \arabic{page} of \protect\pageref*{LastPage} % +LP
%                    \arabic{page}                               % -LP
                    \hfill \,}}

% Packages used for icons. FA icons start with \fa
\usepackage{bbding}
\usepackage{fontawesome5}

% Finally, give us PDF bookmarks
\usepackage{color,hyperref}
\definecolor{darkblue}{rgb}{0.0,0.0,0.3}
\hypersetup{colorlinks,breaklinks,
            linkcolor=darkblue,urlcolor=darkblue,
            anchorcolor=darkblue,citecolor=darkblue}

%%%%%%%%%%%%%%%%%%%%%%%% End Document Setup %%%%%%%%%%%%%%%%%%%%%%%%%%%%


%%%%%%%%%%%%%%%%%%%%%%%%%%% Helper Commands %%%%%%%%%%%%%%%%%%%%%%%%%%%%

% The title (name) with a horizontal rule under it
% (optional argument typesets an object right-justified across from name
%  as well)
%
% Usage: \makeheading{name}
%        OR
%        \makeheading[right_object]{name}
%
% Place at top of document. It should be the first thing.
% If ``right_object'' is provided in the square-braced optional
% argument, it will be right justified on the same line as ``name'' at
% the top of the CV. For example:
%
%       \makeheading[\emph{Curriculum vitae}]{Your Name}
%
% will put an emphasized ``Curriculum vitae'' at the top of the document
% as a title. Likewise, a picture could be included:
%
%   \makeheading[\includegraphics[height=1.5in]{my_picutre}]{Your Name}
%
% the picture will be flush right across from the name.
\newcommand{\makeheading}[2][]%
        {\hspace*{-\marginparsep minus \marginparwidth}%
         \begin{minipage}[t]{\textwidth+\marginparwidth+\marginparsep}%
             {\large \bfseries #2 \hfill #1}\\[-0.15\baselineskip]%
                 \rule{\columnwidth}{1pt}%
         \end{minipage}}

% The section headings
%
% Usage: \section{section name}
\renewcommand{\section}[1]{\pagebreak[3]%
    \hyphenpenalty=10000%
    \vspace{1.3\baselineskip}%
    \phantomsection\addcontentsline{toc}{section}{#1}%
    \noindent\llap{\scshape\smash{\parbox[t]{\marginparwidth}{\raggedright #1}}}%
    \vspace{-\baselineskip}\par}

% An itemize-style list with lots of space between items
\newenvironment{outerlist}[1][\enskip\textbullet]%
        {\begin{itemize}[#1,leftmargin=*]}{\end{itemize}%
         \vspace{-.6\baselineskip}}

% An environment IDENTICAL to outerlist that has better pre-list spacing
% when used as the first thing in a \section
\newenvironment{lonelist}[1][\enskip\textbullet]%
        {\begin{list}{#1}{%
        \setlength{\partopsep}{0pt}%
        \setlength{\topsep}{0pt}}}
        {\end{list}\vspace{-.6\baselineskip}}

% An itemize-style list with little space between items
\newenvironment{innerlist}[1][\enskip\textbullet]%
        {\begin{itemize}[#1,leftmargin=*,parsep=0pt,itemsep=0pt,topsep=0pt,partopsep=0pt]}
        {\end{itemize}}

% An environment IDENTICAL to innerlist that has better pre-list spacing
% when used as the first thing in a \section
\newenvironment{loneinnerlist}[1][\enskip\textbullet]%
        {\begin{itemize}[#1,leftmargin=*,parsep=0pt,itemsep=0pt,topsep=0pt,partopsep=0pt]}
        {\end{itemize}\vspace{-.6\baselineskip}}

% To add some paragraph space between lines.
% This also tells LaTeX to preferably break a page on one of these gaps
% if there is a needed pagebreak nearby.
\newcommand{\blankline}{\quad\pagebreak[3]}
\newcommand{\halfblankline}{\quad\vspace{-0.5\baselineskip}\pagebreak[3]}

% Uses hyperref to link DOI
\newcommand\doilink[1]{\href{http://dx.doi.org/#1}{#1}}
\newcommand\doi[1]{doi:\doilink{#1}}

% For \url{SOME_URL}, links SOME_URL to the url SOME_URL
\providecommand*\url[1]{\href{#1}{#1}}
% Same as above, but pretty-prints SOME_URL in teletype fixed-width font
\renewcommand*\url[1]{\href{#1}{\texttt{#1}}}

% For \email{ADDRESS}, links ADDRESS to the url mailto:ADDRESS
\providecommand*\email[1]{\href{mailto:#1}{#1}}
% Same as above, but pretty-prints ADDRESS in teletype fixed-width font
%\renewcommand*\email[1]{\href{mailto:#1}{\texttt{#1}}}

%\providecommand\BibTeX{{\rm B\kern-.05em{\sc i\kern-.025em b}\kern-.08em
%    T\kern-.1667em\lower.7ex\hbox{E}\kern-.125emX}}
%\providecommand\BibTeX{{\rm B\kern-.05em{\sc i\kern-.025em b}\kern-.08em
%    \TeX}}
\providecommand\BibTeX{{B\kern-.05em{\sc i\kern-.025em b}\kern-.08em
    \TeX}}
\providecommand\Matlab{\textsc{Matlab}}

%%%%%%%%%%%%%%%%%%%%%%%% End Helper Commands %%%%%%%%%%%%%%%%%%%%%%%%%%%

%%%%%%%%%%%%%%%%%%%%%%%%% Begin CV Document %%%%%%%%%%%%%%%%%%%%%%%%%%%%

\begin{document}
\makeheading{Nehemias Ulloa}

\section{Contact Information}

% NOTE: Mind where the & separators and \\ breaks are in the following
%       table.
%
% ALSO: \rcollength is the width of the right column of the table
%       (adjust it to your liking; default is 1.85in).
%
\newlength{\rcollength}\setlength{\rcollength}{1.4in}%
%
\begin{tabular}{@{}p{2in}p{4in}}
\Phone \: 661.428.5419 & \faEnvelope[regular] \: nehemiasulloa@gmail.com \\            
\faGlobe \: \href{https://nulloa.github.io/}{nulloa.github.io} & \faLocationArrow \: Chula Vista, CA \\   
\end{tabular}
%\section{Objective}

%Insert text here if you want to
%\begin{innerlist}
%\item More information and auxiliary documents can be found at\\\url{http://www.tedpavlic.com/facjobsearch/}
%\end{innerlist}

% \section{Position}
% Statistician, Data Scientist


%%%%% Work Experience %%%%%
\section{Work Experience}

\textbf{Center for Veterinary Biologics (CVB), APHIS, USDA}  \hfill {\textbf{July 2019 - Present}}\\
Remote - Chula Vista, CA, Office - Ames, IA\\
\textbf{Senior Staff Statistician} \hfill {\textbf{Nov 2022 - Present}}
\begin{innerlist}
  \item Collaborated with scientists to create regulations for new technologies
  \item Evaluated expeimental designs and their datasets from vaccine safety and efficacy trials
  \item Consulted with researchers to provide statistical advice on their projects
  \item Filled in as acting supervisor when needed
  \item Oversaw implementation of shiny applications and a shiny server \\
\end{innerlist}

\textbf{Staff Statistician} \hfill {\textbf{July 2019 - Nov 2022}}
\begin{innerlist}
  \item Evaluated over 400 expeimental designs and their datasets from vaccine safety and efficacy trials
  \item Provided statistical advice on sound study design to regulators, scientists and researchers
  \item Managed multiple projects related to statistical evaluation and consultation \\
\end{innerlist}

\textbf{Iowa Monarch Conservation Consortium} \hfill {\textbf{Jan 2019 - June 2019}} \\
Ames, IA\\
\textbf{Research Assistant}
\begin{innerlist}
  \item Maintained and cleaned datasets for a \verb|R| package
  \item Created figures and tables for grant reports \\
\end{innerlist}

\textbf{After, Inc.} \hfill {\textbf{May 2018 - Dec 2018}} \\
Norwalk, Connecticut\\
\textbf{\texttt{R} Shiny Developer Intern}
\begin{innerlist}
  \item Developed client reports with \verb|R| Shiny in collaboration with other statisticians
  \item Createed new and improved reporting templates to standardize reports across clients
  \item Researched new Shiny features for new Shiny apps and documents \\
  % \item Hours: 40+ hrs/week during summer, 20 hrs/week during the semester \\
  % \item Reference: Dan Adelsberg, (203) 254-5324, dadelsberg at afterinc.com (Contact: yes)\\
\end{innerlist}

\textbf{Iowa State University} \hfill {\textbf{Aug 2013 - May 2018; Jan 2019 - June 2019}} \\
Ames, IA \\
\textbf{Instructor/Teaching Assistant}
\begin{innerlist}
  \item Taught introductory statistics both in person and online 
  % \item Courses: STAT 101 and STAT 305 (Engineering Statistics) \\
  % \item Hours: 20 hrs/week \\
  \item Ran labs and graded for both undergraduate and graduate courses 
  % \item Held office hours to help students with a one-on-one emphasis
  % \item Graded homework and lab assignments
  % \item Reference: Jarad Niemi, (515) 294-8679, niemi at iastate.edu; Kevin Kasper, kmkasper at iastate.edu; Petrutza Caragea, (515) 294-5582, pcaragea at iastate.edu (Contact: yes to all)\\
  % \item Courses: STAT 101 (Principles of Statistics), STAT 326 (Introduction to Business Statistics II), STAT 330 (Probability and Statistics for Computer Science), STAT 401 (Statistical Methods for Research Workers), STAT 544 (Graduate-level Bayesian Statistics)
\end{innerlist}

% \textbf{Instructor}
% \begin{innerlist}
%   \item Lectured a section of introductory statistics one semester (class size $\sim$ 100) and taught online one semester
%   \item Courses: STAT 101 and STAT 305 (Engineering Statistics) \\
%   % \item Hours: 20 hrs/week \\
%   % \item Reference: Jarad Niemi, (515) 294-8679, niemi at iastate.edu; Kevin Kasper, kmkasper at iastate.edu; Petrutza Caragea, (515) 294-5582, pcaragea at iastate.edu (Contact: yes to all)\\
% \end{innerlist}

% \textbf{California State University, Bakersfield} \hfill {\textbf{Summer 2013}}\\
% Bakersfield, CA \\
% % \textbf{Math Department Tutoring Center}
% % \begin{innerlist}
% %   \item Tutored students from algebra through the calculus series
% %   % \item Hours: 10-20 hrs/week during the quarter
% %   % \item Reference: Charles Lam, (661) 654-2403, clam at csub.edu (Contact: yes)
% % \end{innerlist}
% 
% \textbf{Instructor}
% \begin{innerlist}
%   \item Ran hybrid course using an interactive online learning environment to help develop basic math skills
%   \item Courses: MATH 75 (Developmental Mathematics I ) \& MATH 85 (Developmental Mathematics II)
%   % \item Hours: 40 hrs/week during the quarter
%   % \item Reference: Terran Felter, (661) 654-6835, tfelter at csub.edu (Contact: yes)\\
% \end{innerlist}


%%%%% Education %%%%%
\section{Education}

{\textbf{PhD in Statistics}}  \hfill {\textbf{June 2019}} \\
Iowa State University

\textit{Thesis:} Bayesian Hierarchical Modeling and Analysis for Disease Outbreaks

\textit{Advisor:} Dr. Jarad Niemi\\

{\textbf{MS in Statistics}} \hfill {\textbf{May 2017}} \\
Iowa State University

\textit{Creative Component:} Application of Polynomial Regression to Dyadic Data

\textit{Advisor:} Dr. Fred Lorenz \\

{\textbf{BS in Mathematics}} \hfill {\textbf{May 2013}} \\
California State University, Bakersfield



%%%%% Research Experience %%%%%
% \section{Research Experience}
% \textbf{CREST(NSF)} \hfill {\textbf{June 2012 - June 2013}}\\
% California State University, Bakersfield
% \begin{innerlist}
%   \item Investigated methods of fitting a penalized broken cubic spline and compared using AIC, AICcorr, BIC, CV, and GCV criteria’s to the RSE\\
% \end{innerlist}
% %
% \textbf{Interdisciplinary Program in High Performance Computing}, \hfill {\textbf{Summer 2012}}\\
% University of Maryland, Baltimore County
% \begin{innerlist}
%   \item Completed Math 447: Introduction to Parallel Computing where I learned parallel computing using C with MPI, R with Snow, and Matlab
%   \item Collaborated with the Department of Natural Resources of Maryland to identify trouble areas of the Chesapeake Bay using different statistical methods and ranking systems.
% \end{innerlist}


% %%%%% Publications %%%%%
% \section{Publications}
% 
% Eduardo L. Montoya, Nehemias Ulloa, and Victoria Miller. (2014) A Simulation Study Comparing Knot Selection Methods With Equally Spaced Knots in a Penalized Regression Spline. \textit{International Journal of Statistics and Probability}, v.3, n.3, p. p96. (\href{http://www.ccsenet.org/journal/index.php/ijsp/article/view/36992}{URL}) \\
% 
% Rosemary K. Le, Christopher V. Rackauckas, Anne S. Ross, and Nehemias Ulloa. (2013) Assessment of Statistical Methods for Water Quality Monitoring in Maryland’s Tidal Waterways. \textit{SIAM Undergraduate Research Online (SIURO)}, vol. 6. (\href{https://www.siam.org/students/siuro/vol6/S01207.pdf}{URL}) \\


%%%%% Awards %%%%%
% \section{Awards}
% \textbf{Alliance for Graduate Education and the Professoriate Fellowship} \hfill {\textbf{Aug 2013 - June 2018}}\\
% Iowa State University
% \begin{innerlist}
%   \item Joint-fellowship between Iowa Universities dedicated to increasing the number of underrepresented minorities obtaining graduate degrees in STEM.
% \end{innerlist}


%%%%% Presentations %%%%%
% \section{Activities}
% \textbf{STAT-ers}: Statistics graduate student organization \hfill {\textbf{Aug 2013 - June 2019}}\\
% Iowa State University
% \begin{innerlist}
%   \item Vice President \hfill {\textbf{2015 - 2016}}
%   \item President \hfill {\textbf{2016 - 2017}} \\
% \end{innerlist}
% 
% \textbf{StatCom}: Statistics in the Community \hfill {\textbf{Aug 2015 - June 2019}}\\
% Iowa State University
% \begin{innerlist}
%   \item Ran statistics station at Meeker Elementary's Science Night
%   \item Helped a non-profit with data clean-up and summary statistics
% \end{innerlist}



%%%%% Computer Skills %%%%%
\section{Computer \\Skills}
\textbf{Programming}:  \verb|R|, C++, SPSS, MPLUS, SQL, Git, \LaTeX \\
\textbf{Statistical Computing}: SAS, Stan, JAGS, Shiny, JMP 


% \section{Professional Memberships}
% American Statistical Association

\section{Languages}
English, Spanish

\section{Personal Interests}
For enjoyment outside of statistics, I enjoy soccer, bike riding, and basketball.


\end{document}

% \section{Education/ Research Background}
% I am in the final year of my Ph.D. program in Statistics at Iowa State University. The core courses I've taken include 4 courses in methods and 5 in theory. My elective courses covered Bayesian statistics, environmental statistics, spatial statistics, and machine learning. My course background gives me a large breadth of knowledge of the principles and methods of statistics and data science. I have experience in modern visualization methods as a Shiny Developer. My internship, TA work and elective projects have given me the opportunity to work with experts in different fields such as Agriculture, Veterinary Science, and Natural Resource Ecology and Management which has refined my skills in the application of statistics and data science to scientific studies. \\
% 
% My research uses Bayesian hierarchical models to study and forecast the influenza season; I implement sparse supervised-learning techniques, hierarchical non-linear models, and incorporate multiple data sources in a Bayesian framework. I've implemented my analysis in \verb|R| and created \verb|R| packages of my code stored on Github for version control and distribution which demonstrates my skills in software development and software application and maintenance. I am a part of the Center for Disease Control's influenza forecasting competition. I collaborated with another graduate student in Natural Resource Ecology and Management on creating a Bayesian classification model for fish eggs which provided me an opportunity to provide consulting services for a scientist. Our model used fish egg characteristics to try and classify the fish type before they hatched in an effort to asses species invasion which required me to gain knowledge of biological concepts and skill in their quantitative application. \\
% 
% 
% 
% 
% \section{References}
% 
% Dr. Guy
% \begin{innerlist}
% \item[] Mayo Professor in Public Health, Division Head \hfill {Phone: 612-624-6646}\\
% Division of Biostatistics \hfill{E-mail: email@gmail.com}\\
% University of Minnesota
% \end{innerlist}
% 
% \halfblankline
% 
% Sudipto Banerjee
% \begin{innerlist}
% \item[] Professor \hfill {Phone: 612-624-0624}\\
% Division of Biostatistics \hfill{E-mail: baner009@umn.edu}\\
% University of Minnesota
% \end{innerlist}


